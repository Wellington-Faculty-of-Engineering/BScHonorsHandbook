\chapter{Final Report}
The final report constitutes the most important component of the
individual project.  This is where you will set out what exactly it is
you have done, why you have done it and how it can improve things.


\section{Format}
The following points clarify the main requirements of the final
report:
\begin{itemize}
\item The report should be written using the ECS report templates
  provided (available for latex and MS Word).  Fonts should be no
  smaller than 11pt.
\item The report must contain a table of contents.
\item The report is expected to be between 10,000 and 15,000 words with an 
absolute limit of 20,000 words (this is about 40 pages). For AIML 487 the expectation is 12,000 - 18,000 words with a limit of 25,000 words. 
Reports which are longer than this will need to be justified to the
supervisor and the course coordinator, or risk being penalised for
excessive length.

\item The report should be written in such a way that any 4th year student in your specialisation
  can understand.  Since the report will be assessed by a panel of
  examiners (i.e. not just the supervisor), it is critical that all
  examiners can properly understand what has been achieved.

\item Material from the preliminary report and/or project proposal may be
  used directly in the final report.
\end{itemize}
The final report must be submitted via the online submission system on or
before the given due date. Extensions will be granted only in exceptional circumstances. These
\emph{must} be arranged in advance through discussion with project supervisors
{\bf and the course coordinator}. It is the students responsibility to ensure the course coordinator is informed of the extension request. 

Take some care with the format of your final document. Remember that we often print the document and you can easily make that very hard for us. Here
are some things to think about:
\begin{itemize}
  \item We do not accept word files. Make a pdf for submission.
  \item Try to use vector graphics (ideally eps or pdf), rather than raster
  formats (jpg, png etc.).
  Not only will this look better it will produce a smaller file that will be
  easier to print.
  \item You do \emph{not} need to use super high resolution graphics. Our
  printer can't reproduce them anyway, so anything greater than 300dpi or so is
  a waste.
  \item Don't use some strange printer driver.
\end{itemize}

\section{Suggested Organisation}
The structure of your report should be tailored to your project. However, a
sensible outline for the final report is as follows:
\begin{itemize}
\item {\bf Introduction.}  The purpose here is to introduce the
  research question and motivate why it is a problem we should
  care about, and to outline process for finding a solution.  {\em Remember}: the introduction is the first part of the
  report an examiner will read. If he/she finishes reading it without
  a proper understanding of the research problem or what has been
  done, then they will almost certainly struggle with understanding
  the remainder. You should attempt to make the research goals and associated
  specifications as clear and as quantifiable as possible. These goals and
  specifications should inform everything else that follows, so it is important
  to establish them in the examiners mind.

\item {\bf Background / Related Work.}  The background should cover
  any important terminology and/or concepts used in the remainder of
  the report, and should demonstrate an understanding of previous
  works which are relevant.  {\em Remember:} A good related work
  section does not just provide a list of previous works, accompanied
  with short summaries.  Wherever possible it must extract real
  insight from these works, painting a picture of how they relate to
  each other and the project.

\item {\bf Methodology.} The aim here is to explain the process of the research. 

\item {\bf Implementation (frequent).} What and how the artifact was created. 

\item {\bf Results.}
  The results from experiment or analysis.
  Make liberal use of graphs and other figures. They are much more effective at
  communicating many results than are words.

\item {\bf Discussion.}
  What do the results mean?

\item {\bf Conclusions and Future Work.}
  Future work should \emph{not} just be a list of things that you would have
  done if you had a little more time. Talk about new things that are possible
  now that you have finished your project. What projects could a '489 student
  tackle next year if they started from your end point?

\item {\bf Bibliography.}
\end{itemize}


\section{Assessment}
The primary purpose of the final report is to clearly and succinctly
detail the design, implementation and evaluation of any artifact
developed.  The report should be written in a professional nature, as
appropriate for the discipline and degree.\\

\subsection{Process}
\noindent The final report will be read by two examiners,
one of whom is the primary supervisor.  A third marker will be included if there is a significant divergence in the marks awarded.  {\bf
  Examiners must complete their marking in a timely fashion, so that
  the committee can meet and determine a final grade for the student}.
In determining the final grade, the examining committee may take into
consideration those (indicative) grades awarded for other assessment
items.

\newpage

\subsection{Criteria}

\noindent The final report will be assessed using the following
criteria:
\begin{itemize}
\item {\bf Quality of work.} This should include but not limited to

\begin{itemize}
  \item {\bf Proposal}  {\em Does the report
    clearly identify the problem being solved, and motivate the reason
    a solution would be valuable?}  Emphasis is placed on connection with existing academic research problems.

  \item {\bf Context}  {\em Does the report provide clear evidence of understanding the previous and current research?}  This includes the coverage and justification for including previous research and the connection to the research question.

  \item {\bf Methodology}  {\em Does the report clearly describe the process used to explore the research question ?}  This section justifies and explains the process used to generate the results in a later section.
  
   \item {\bf Implementation (often)}  {\em Does the report describe the artifact created and relevant decisions relevant to the research?}  This section is for students which create something to test a theory. This could be software, a framework, mathematical model or proof. 

  \item {\bf Results.}  {\em Does the report provide a clear set of results from the experimentation/evaluation?}    This should report on the data collected and present it in a coherent fashion. This would likely include statistical tests and figures to help the reader understand the data. 

  \item {\bf Discussion and Conclusion.}  {\em Does the report provide discuss the importance and limitations of the results and experiment in general?}  There should be both an evaluation of the results and a section on the limitations of the project. The Conclusion should include both a reflection on the overall meaning of the research and a future work section. 

  \item {\bf Critical Thinking.}  {\em Does the report provide clear
    evidence of critical thought?}  This should be evident throughout
  the report, with an emphasis on the discussion section.
  \end{itemize}

\item {\bf Presentation.}  {\em Is the report written in an
    appropriate and professional manner, with due consideration given
    to presentation?}  This includes, but is not limited to: overall
  report structure; spelling and grammar; consistent bibliography
  layout including all necessary information (e.g. journal/conference
  title, page numbers, year, author names, article title);
  presentation and layout of figures and tables; minimum requirements
  of written English.

\end{itemize}
These criteria are, by definition, subject to the examiner's
individual interpretation.  In any case where an examiner is uncertain
regarding some aspect of the criteria or process, the course
coordinator should be consulted.

Among the current BSc Hons, only AIML487 has 45 points, and its main difference from the projects with 30 points is the workload. With the additional 150 hours, the AIML487 students are expected to have more work done, and have some novelty and original contributions in the research project. Since the number of students in AIML487 is small in 2021, the project can be handled case by case. The course coordinator should be consulted if an examiner is uncertain regarding the grade. 
