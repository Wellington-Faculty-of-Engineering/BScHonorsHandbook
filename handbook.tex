%% $RCSfile: proj_report_outline.tex,v $
%% $Revision: 1.2 $
%% $Date: 2010/04/23 02:40:16 $
%% $Author: kevin $

\documentclass[11pt
              , a4paper
              , oneside
              ]{report}


\usepackage{float} % lets you have non-floating floats
\usepackage{multicol}
\usepackage{hyperref}
\hypersetup{
    colorlinks=true,
    linkcolor=blue,
    filecolor=magenta,      
    urlcolor=cyan,
}
\usepackage{url} % for typesetting urls
\usepackage[usenames,dvipsnames,svgnames]{xcolor}
\usepackage{color}
\definecolor{darkgreen}{rgb}{0,0.6,0}
%\newcommand{\bryan}[1]{\textcolor{darkgreen}{[BN: #1]}}

%
%  We don't want figures to float so we define
%
\newfloat{fig}{thp}{lof}[chapter]
\floatname{fig}{Figure}

%% These are standard LaTeX definitions for the document
%%

\title{{\Huge BSc Hons Handbook
2022}\vspace{1cm}\\ ELCO489, CGRA489, COMP489, and AIML487 \\

(Individual Research Project)}

%% Original authors of the document, David J. Pearce, Christopher Hollitt
%% Edition, 2014 Sharon Gao and Kris Bubendorfer, added new sections on
%% the allocation process
%% Edition, 2015 Kris Bubendorfer, added chapter on supervision
%% Edition 2019 Ramesh Rayudu, added on industry projects, changed all Bryan's highlights, changed COMP489 description
%% Edition 2020 Ramesh Rayudu, changed dates
%% Edition 2021 Simon McCallum, converted to more research and less engineering
%% Edition 2022 Simon McCallum, updating content including word limits and ethics.

%% This file can be used for creating a wide range of reports
%%  across various Schools
%%
%% Set up some things, mostly for the front page, for your specific document
%
% Current options are:
% [ecs|msor]              Which school you are in.
%
% [bschonscomp|mcompsci]  Which degree you are doing
%                          You can also specify any other degree by name
%                          (see below)
% [font|image]            Use a font or an image for the VUW logo
%                          The font option will only work on ECS systems
%
\usepackage{vuwhandbook}

% You should specifiy your supervisor here with
%     \supervisor{Firstname Lastname}
% use \supervisors if there is more than one supervisor

% Unless you've used the bschonscomp or mcompsci
%  options above use
%   \otherdegree{OTHER DEGREE OR DIPLOMA NAME}
% here to specify degree

% Comment this out if you want the date printed.
\date{}

% creating new commands as macros
\newcommand{\bsc}{{\bf COMP}, {\bf CGRA}, or {\bf ELCO }}
\newcommand{\ai}{{\bf AIML}}

% crating new commands for commenting
\newcommand{\n}[1]{{\color{green}{#1}}}
% renewcommand{\n}[1]{} % this removes all notes from the pdf
\newcommand{\dn}[1]{} % this is used for notes and are done.

\begin{document}

% Make the page numbering roman, until after the contents, etc.
\frontmatter

%%%%%%%%%%%%%%%%%%%%%%%%%%%%%%%%%%%%%%%%%%%%%%%%%%%%%%%

%%%%%%%%%%%%%%%%%%%%%%%%%%%%%%%%%%%%%%%%%%%%%%%%%%%%%%%

 \begin{abstract}

   The ELCO489, CGRA489, and COMP489 are an individual research
   project which is a major undertaking for both students and staff.  This document
   describes in detail the requirements for these courses, and provides some
   additional guidelines.

   Staff expect that students are completely familiar with the contents
   of this document.  Questions regarding this document should be
   discussed with the supervisor (where appropriate) or the course
   coordinator (otherwise).

 \end{abstract}

%%%%%%%%%%%%%%%%%%%%%%%%%%%%%%%%%%%%%%%%%%%%%%%%%%%%%%%

\maketitle

\tableofcontents

%%%%%%%%%%%%%%%%%%%%%%%%%%%%%%%%%%%%%%%%%%%%%%%%%%%%%%%

\mainmatter

%%%%%%%%%%%%%%%%%%%%%%%%%%%%%%%%%%%%%%%%%%%%%%%%%%%%%%%

% individual chapters included here
\chapter{Introduction}
The BSc Hons (COMP/ELCO/CGRA489 and AIML487) project courses consist of an individual or small 
group project
done under the supervision of one (or more) academic staff.  Some projects
are also offered in partnership with industry - in which case
supervision is shared with an industry supervisor.     The underlying
aim of the project is to develop your research ability and demonstrate your independent  
and critical thinking.  The project will involve refining a project proposal, reviewing and analysing relevant literature, potentially developing an artifact, and evaluating the artifact or research problem.  
You will present a final written report, and conclude with
an oral presentation.

\section{Aims and Scope}
The aim of this document is to provide a comprehensive guide to the BSc Hons  project, for both students and staff. In
particular, the document sets out the requirements of the course and clarifies the way in which student projects 
will be assessed and supervised.

\section{Engineering versus Science}
Many of the students in 400 level courses are doing ENGR489, that is an engineering project which is more about implementation and process, while this handbook describes the
\textbf{research} projects.

Research projects at 4th year range from literature reviews and analysis, repeating existing published work, through to the testing of new artifacts, and creation of knowledge.  The first part of the project is to define the scope of the project and questions that are being answered.
If an artifact is created then there needs to be an experimental protocol to collect, 
analyse and interpret data. BSc Hons students are expected to
  demonstrate mathematical rigour (where appropriate), and use
  scientific experimentation to make critical observations. The literature
  survey for the projects will typically draw on research papers in journals
  and conferences.

{\bf NOTE:} If unsure you should consult with their supervisor(s) and/or
the course coordinator to ensure that the project is a research project.

\section{Plan, Create, Test and Evaluate}

A project can be thought of as planning, reviewing the research area, typically creating an artifact based on research,
and evaluating the {\em artifact}.  The term artifact
refers to student created software, theoretical framework, taxonomy, dataset or other work apart from than the report created as part of the project. 

\pagebreak
\section{Project Timeline}
The following provides a rough overview of the project timeline, and identifies the main points of interest.
\begin{center}
\begin{tabular}{|l|p{10cm}|}
\hline
Week 1 & {Students rank projects using project allocation system.}\\
Week 2 & Project allocation performed by course coordinator.\\
Week 3 & Students meet with supervisor(s) and begin work.\\
Week 4 & \textbf{ students submit project proposals} and an  IP forms on ECS submission system\\
week 5 & Work continues; students meet regularly with supervisors.\\
...&  $\ldots$\\
Week 12 & \textbf{Students submit their progress report.} Thus is worth 20\% is it increases the final grade.\\
\hline
\hline
Mid-Year Break& Work continues around examinations.\\
&Students meet with supervisors where possible.\\
\hline
\hline
Week 1 & {Students give a presentation to their associated research group}\\
...& Work continues; students meet regularly with supervisors.\\
Week 7 (end of)& {Students submit a draft of their final report to their Supervisors.}\\
Week 12 (end of)& \textbf{Students submit final report.}\\
\hline
\hline
T2 Exam Period & \textbf{Students present their work during conference day.}\\
\hline
\end{tabular}
\end{center}

\chapter{Project Allocation}
The first stage of the BSc Hons project is the
allocation of projects to students.  This process attempts to allocate
students to the projects they prefer.  Indeed, it is in the interests
of both students and staff that this is done as accurately, and
quickly as possible.  Once the allocation is complete, students must
produce a project proposal in conjunction with their
supervisor(s).

\section{Choosing a Project}

Projects are listed on the \href{https://ecs.wgtn.ac.nz/Courses/COMP489_2022FY/Projects}{COMP 489 Projects page} linked from the combined course page currently listed under COMP 489. There will be a link on this page for the form to submit your preferences for projects.  
You should generally pick projects you have talked to the supervisor about.

We cannot guarantee that every student will be allocated to a project they
prefer.  In the unlikely event of a student being allocated to a
project that they believe is not suitable for them, they should
immediately contact the course coordinator.

\section{Proposal}

Once the allocation of students to projects is complete, students are
expected to meet with their supervisors and put together a {\em project proposal}.

\noindent {NOTE:} it is the student's responsibility to contact their
supervisor and arrange an appropriate meeting time.\\
{Students are required to submit a report (no more than four pages) for the proposal stage by the end of week 5. Generally,
the report should include the following topics:}

\begin{itemize}
\item An overview of the research problem being addressed by the project.
\item A high-level summary of state-of-the-art techniques to solve this problem.
\item A statement of key motivations, including limitations or issues that the current/state-of-the-art methods have and this project is to tackle.
\item A statement of the overall goal and specific objectives, hypotheses, or research questions.
\item A statement regarding the proposed method to investigate to the problem.
\item A statement regarding the proposed evaluation method, e.g. possible available data and the evaluation measures.
\item A discussion of any ethical considerations around the project.
\item A statement, if applicable, regarding any budgetary requirements, including
  appropriate justification.
\item A statement regarding any risks or hazards that the project
  poses (either in the development itself, or in using the final
  artifact).
\item A discussion of any other requirements for the project to be successfully
completed. This might be access to particular equipment or rooms, special IP
issues etc.
\item Provide a proposed project time line, in the form of a Gantt chart (or similar).
\item Provide a plan B to continue working on the project if Covid 19 restrictions cause limited access to equipment, environment, or users.
\end{itemize}


{Proposal pages not included in the page limit:}
\begin{itemize}
	\item Title Page
	\item Table of Contents and Glossary
	\item References and Bibliography
	\item Project Gantt Chart
\end {itemize}

There is the potential to provide some funding for some projects.  
The funding is primarily to help purchase items
necessary for the project, although it can be used for other purposes
(e.g. as koha for user-experiments or surveys).  Students must justify their
budgetary requirements in the proposal report.

For external projects, it is the norm that the sponsoring organisation fund any related costs for the project. Any exceptions will need an approval from the Head of School.

\subsection{Assessment Process}

{Constructive feedback should be given two weeks after the report submission deadline.  The examiners are expected to read the report and give feedback to the supervisors.}

The aim of this process is to identify:
firstly, whether the project is viable and sensible for the given
specialisation; secondly, whether there are any obvious issues which
must be addressed.  Where necessary, some comments will be communicated
to the student by the supervisor.

\section{Intellectual Property Agreement}

All students working with external partners are required to submit a signed intellectual property agreement along with their {proposal report}.  The purpose of the intellectual property agreement is simply to identify the stakeholders in the project.

\chapter{Supervision and You}

%% This text is based on the regulations for PhD supervision from
%% FGR.  This has been simplified and repackaged for Hons.

As this is likely your first experience with an individual supervised project, it
can be difficult to calibrate your expectations against your
supervisor's.  This section aims to describe what you
should be getting from your supervisor, and what your supervisor
should be getting from you.  If your experience under supervision differs widely from
the guidelines given in this chapter, you should first discuss this with
your supervisor and if it cannot be resolved to your satisfaction
please bring your query to the BSc Hons coordinator.  The earlier issues are
identified and resolved, the better things will be handled.  There is
little we can do to resolve long standing problems a few weeks out
from submission.

\section{Supervisors Responsibilities}

It is the responsibility of your supervisors to guide you through the
academic requirements of your project. Supervisors will:

\begin{itemize}
\item meet with you regularly
\item provide you with academic guidance and scholarly direction
\item assess your progress and give you feedback
\item act as guides to University facilities
\item make sure you comply with the University’s administrative regulations.
\end{itemize}

\subsection{Regular Meetings}

It is expected that you and your supervisor will meet in person
regularly and individually:

\begin{itemize}
\item For a project supervised by a VUW staff member we would expect 
a 30 minute individual meeting each week.  You may agree
with your supervisor to different arrangements that better suit the
nature of the project, but it important to have regular meetings.

\item For a project supervised by an external supervisor and VUW
supervisor, we would expect a fortnightly supervision meeting with the
external supervisor.  The VUW supervisor would not be expected to
attend every meeting, although may, but would generally have a combined meeting at
least every month.
\end{itemize}

\subsection{Academic Guidance}

Your supervisors in conjunction with the research seminars and course coordinators 
will provide guidance on a range of academic matters. These include:

\begin{itemize}
\item the standards required for an honours project
\item planning your research
\item skills you will need to acquire
\item research resources
\item methodology
\item undertaking a literature review
\item ethical, legal, professional and safety issues
\end{itemize}

Throughout, your supervisors will bear in mind the expectations of examiners.

\subsection{Assessing Your Progress and Feedback}

Supervisors will assess your progress and provide you with constructive feedback throughout your project. 
They will need to ensure that you possess the understanding and abilities to:

\begin{itemize}
\item carry out your project as envisioned
\item complete your work on time, meeting the various deadlines for assessment.
\item Provide prompt feedback on your work.  The university guideline
  for feedback is 3 weeks, ECS aims for a 2 week turnaround.
\end{itemize}

\subsection{Support}

Your supervisor(s) is also there to support you. If you encounter
problems of any kind, you should feel free to discuss them with them -
especially if it could have an impact on your project work.  The BSc Hons 
coordinator is also available to help and offer support in such
situations, especially if you are not comfortable discussing matters
directly with your supervisor.  If they can't help, they will be able
to direct you to various student support services run by the
University -- a guidline to these services will be linked from the
course homepage.

\section{Your Responsibilities as a Project Student}

You will need to abide by the University regulations governing your
degree.

\subsection{Planning and Actively Pursuing Your Work}

You have an obligation to devote sufficient time to your work, to
complete each phase on time, and to avoid activities that interfere
with your satisfactory and timely completion of the project.  
For the 30 point courses: CGRA489, COMP489 and ELCO489
they are 15 points per trimester, thus you should expect to spend 
on average 10 hours per week on your project,
spread over the 30 weeks that the course runs (i.e. including
mid-trimester breaks, and the mid-year break).  
For the 45 point course AIML487, you will likely have one semester with 
more work, thus in your semester with only 2 other courses you are likely to work 20 hours per week.
It can be quite challenging to maintain steady progress and dedicate the time as
course loads increase during the trimester, however, it is important
that you manage your time well so that you can devote time each week to the project.  
You will get little benefit from your
supervisor if you treat your project as a series of short term crunches.
Certainly they will not be able to provide timely feedback or
appropriate guidance in this situation.

\subsection{Ethics}

It is expected that you conduct your research in an ethical
manner.  We already have ethical approval for standard
user interface testing for software Ethics \#29386 linked from the course resources page
Additionally you must:
\begin{itemize}
\item where appropriate, discuss ethics with your supervisors
\item familiarise yourself with the \href{https://www.wgtn.ac.nz/research/support/ethics}{University’s ethical guidelines}
\item obtain approval from the relevant ethics committee for work involving human or animal subjects.
\end{itemize}
The link to Ethics application process is also available on the BSc Hons Wiki.

It is also important to conduct yourself ethically in your project in relation to 
academic misconduct.  Breaches of the plagiarism or other forms of misconduct
will be treated very seriously. If you are concerned about plagiarism you 
should talk to your supervisor or the course coordinator to clarify the 
situation.

\subsection{Safety \& Health}
The university's approach to health and safety is based on risk management. 
There is a significant strengthening of level of responsibility for students and supervisors. 
Students must discuss with supervisors and show in the project proposal report 
(due at the end of week 5) that they have identified safety risks and developed a plan to manage them.

\noindent Students are expected to be aware of the Health and Safety at Work Act 2015. See : \linebreak  {\footnotesize\url{http://www.business.govt.nz/worksafe/hswa}}.

Students need to discuss with their supervisors and fill out the health and safety 
plan available on the ECS Wiki. They need to fill 'ECS Project Information Form' 
and 'Project Description and Safety Plan'. A sample can be found on the Wiki.

Please Note: For any work that takes place off VUW campuses, the students need fill 
'ECS Off Campus Activity Plan'. Please contact Roger Cliffe for the form.

ALL filled Health and Safety forms must be emailed to \href{mailto://ecs-safety@ecs.vuw.ac.nz}{ecs-safety@ecs.vuw.ac.nz}.

\chapter{Preliminary Report}
At the conclusion of the first trimester, students are expected to
submit a preliminary report which outlines the progress they have made,
and identifies any outstanding issues where feedback is required. This
report should be considered a first step towards the final report -
including a good treatment of the introduction and related/background
work.  However, as a primary purpose of the preliminary report is to give the examination
committee the opportunity to comment on the student's progress (and
identify any areas of concern), it will also include sections on work
done, requests for feedback, and a revised timeline.

{\bf Please Note:} The preliminary report grade is used as part of the final grade 
if it helps increase the overall grade.

\section{Suggested Organisation}
A sensible outline for the preliminary report is as follows:
\begin{itemize}
\item {\bf Introduction / Proposal Review.}  This should briefly
  outline the project and if necessary reevaluate the original plan in light of
  what has been learned in the interim.  In particular, any significant
  deviations in the research problem being addressed
  should be clearly highlighted and justified.
\item {\bf Literature review}.  This should discuss any existing
  solutions to the given problem, and may reference academic papers,
  books and other sources as appropriate.  Care should be taken to
  identify key differences between these solutions, and that being
  developed in the project.
\item {\bf Work Done.}  This should discuss what progress has been
  made.  In many cases, the evaluation proper will not yet have begun.  However,
  it is important to demonstrate that sufficient thought has been
  given to the evaluation.
\item {\bf Future Plan.}  This should highlight the main components
  which remain to be done, and provide a proposed time-line in which
  this will happen.  In putting together a time line, students must
  take into account upcoming examinations, coursework deadlines and
  other disruptions.
\item {\bf Request for Feedback.}  This should highlight any
  difficulties currently faced, and make specific requests for
  guidance from the examination committee.  For example, a student may
  be unsure how best to evaluate their artifact, and would appreciate
  suggestions for alternative methods.
\end{itemize}

The report does not have to confirm to the above structure.
For example, in some cases, students may wish to present preliminary
experimental results, or include a more detailed literature survey, 
particularly if they are doing the 45 point course and so the report
is two thirds of the way through the project rather than halve way.\\

\noindent {\bf NOTE:} in the event of an aegrotat application, the preliminary
report may be used (in conjunction with the snapshot submission) as guidance 
for final grade.

\section{Getting help with writing}
Students struggling with writing and presentation should seek help from the student 
learning support as early as possible.
{\footnotesize \url{http://www.victoria.ac.nz/st_services/slss/}}.


\section{Format}
The following points clarify the main requirements of the preliminary
report:
\begin{itemize}
\item The report should be written using the ECS report templates provided 
on the resources page of the course site.
(available for latex and MS Word).
\item The report is expected to be around 8 pages in length. As a rough
breakdown, a page of introduction and three to four pages on
background/related work.  An additional page each on progress
and future plans would be appropriate. Longer (or shorter) reports are
permitted, but students are advised to ensure all necessary detail is provided.
\item The report should be written in such a way that any 4th year student in your specialisation
  can understand.  Since the report will be assessed by an independent
  examiner (i.e. not just the supervisor), it is critical that all
  examiners can properly understand what has been achieved.
\item The report should include the original project proposal as an
  appendix.
\end{itemize}
Finally, the preliminary report must be submitted via the \href{https://apps.ecs.vuw.ac.nz/submit/}{online
  submission system} on or before the given due date.


\section{Assessment Process}

The preliminary report will be read by two examiners, one of
which is the primary supervisor. Students are required to give a 5 minute
presentation at a specialisation meeting of the primary supervisor.
Constructive feedback should be given after the presentation.
We may record the sessions, so students can reference feedback.



\chapter{Final Report}
The final report constitutes the most important component of the
individual project.  This is where you will set out what exactly it is
you have done, why you have done it and how it can improve things.


\section{Format}
The following points clarify the main requirements of the final
report:
\begin{itemize}
\item The report should be written using the ECS report templates
  provided (available for latex and MS Word).  Fonts should be no
  smaller than 11pt.
\item The report must contain a table of contents.
\item The report is expected to be between 10,000 and 15,000 words with an 
absolute limit of 20,000 words (this is about 40 pages). For AIML 487 the expectation is 12,000 - 18,000 words with a limit of 25,000 words. 
Reports which are longer than this will need to be justified to the
supervisor and the course coordinator, or risk being penalised for
excessive length.

\item The report should be written in such a way that any 4th year student in your specialisation
  can understand.  Since the report will be assessed by a panel of
  examiners (i.e. not just the supervisor), it is critical that all
  examiners can properly understand what has been achieved.

\item Material from the preliminary report and/or project proposal may be
  used directly in the final report.
\end{itemize}
The final report must be submitted via the online submission system on or
before the given due date. Extensions will be granted only in exceptional circumstances. These
\emph{must} be arranged in advance through discussion with project supervisors
{\bf and the course coordinator}. It is the students responsibility to ensure the course coordinator is informed of the extension request. 

Take some care with the format of your final document. Remember that we often print the document and you can easily make that very hard for us. Here
are some things to think about:
\begin{itemize}
  \item We do not accept word files. Make a pdf for submission.
  \item Try to use vector graphics (ideally eps or pdf), rather than raster
  formats (jpg, png etc.).
  Not only will this look better it will produce a smaller file that will be
  easier to print.
  \item You do \emph{not} need to use super high resolution graphics. Our
  printer can't reproduce them anyway, so anything greater than 300dpi or so is
  a waste.
  \item Don't use some strange printer driver.
\end{itemize}

\section{Suggested Organisation}
The structure of your report should be tailored to your project. However, a
sensible outline for the final report is as follows:
\begin{itemize}
\item {\bf Introduction.}  The purpose here is to introduce the
  research question and motivate why it is a problem we should
  care about, and to outline process for finding a solution.  {\em Remember}: the introduction is the first part of the
  report an examiner will read. If he/she finishes reading it without
  a proper understanding of the research problem or what has been
  done, then they will almost certainly struggle with understanding
  the remainder. You should attempt to make the research goals and associated
  specifications as clear and as quantifiable as possible. These goals and
  specifications should inform everything else that follows, so it is important
  to establish them in the examiners mind.

\item {\bf Background / Related Work.}  The background should cover
  any important terminology and/or concepts used in the remainder of
  the report, and should demonstrate an understanding of previous
  works which are relevant.  {\em Remember:} A good related work
  section does not just provide a list of previous works, accompanied
  with short summaries.  Wherever possible it must extract real
  insight from these works, painting a picture of how they relate to
  each other and the project.

\item {\bf Methodology.} The aim here is to explain the process of the research. 

\item {\bf Implementation (frequent).} What and how the artifact was created. 

\item {\bf Results.}
  The results from experiment or analysis.
  Make liberal use of graphs and other figures. They are much more effective at
  communicating many results than are words.

\item {\bf Discussion.}
  What do the results mean?

\item {\bf Conclusions and Future Work.}
  Future work should \emph{not} just be a list of things that you would have
  done if you had a little more time. Talk about new things that are possible
  now that you have finished your project. What projects could a '489 student
  tackle next year if they started from your end point?

\item {\bf Bibliography.}
\end{itemize}


\section{Assessment}
The primary purpose of the final report is to clearly and succinctly
detail the design, implementation and evaluation of any artifact
developed.  The report should be written in a professional nature, as
appropriate for the discipline and degree.\\

\subsection{Process}
\noindent The final report will be read by two examiners,
one of whom is the primary supervisor.  A third marker will be included if there is a significant divergence in the marks awarded.  {\bf
  Examiners must complete their marking in a timely fashion, so that
  the committee can meet and determine a final grade for the student}.
In determining the final grade, the examining committee may take into
consideration those (indicative) grades awarded for other assessment
items.

\newpage

\subsection{Criteria}

\noindent The final report will be assessed using the following
criteria:
\begin{itemize}
\item {\bf Quality of work.} This should include but not limited to

\begin{itemize}
  \item {\bf Proposal}  {\em Does the report
    clearly identify the problem being solved, and motivate the reason
    a solution would be valuable?}  Emphasis is placed on connection with existing academic research problems.

  \item {\bf Context}  {\em Does the report provide clear evidence of understanding the previous and current research?}  This includes the coverage and justification for including previous research and the connection to the research question.

  \item {\bf Methodology}  {\em Does the report clearly describe the process used to explore the research question ?}  This section justifies and explains the process used to generate the results in a later section.
  
   \item {\bf Implementation (often)}  {\em Does the report describe the artifact created and relevant decisions relevant to the research?}  This section is for students which create something to test a theory. This could be software, a framework, mathematical model or proof. 

  \item {\bf Results.}  {\em Does the report provide a clear set of results from the experimentation/evaluation?}    This should report on the data collected and present it in a coherent fashion. This would likely include statistical tests and figures to help the reader understand the data. 

  \item {\bf Discussion and Conclusion.}  {\em Does the report provide discuss the importance and limitations of the results and experiment in general?}  There should be both an evaluation of the results and a section on the limitations of the project. The Conclusion should include both a reflection on the overall meaning of the research and a future work section. 

  \item {\bf Critical Thinking.}  {\em Does the report provide clear
    evidence of critical thought?}  This should be evident throughout
  the report, with an emphasis on the discussion section.
  \end{itemize}

\item {\bf Presentation.}  {\em Is the report written in an
    appropriate and professional manner, with due consideration given
    to presentation?}  This includes, but is not limited to: overall
  report structure; spelling and grammar; consistent bibliography
  layout including all necessary information (e.g. journal/conference
  title, page numbers, year, author names, article title);
  presentation and layout of figures and tables; minimum requirements
  of written English.

\end{itemize}
These criteria are, by definition, subject to the examiner's
individual interpretation.  In any case where an examiner is uncertain
regarding some aspect of the criteria or process, the course
coordinator should be consulted.

Among the current BSc Hons, only AIML487 has 45 points, and its main difference from the projects with 30 points is the workload. With the additional 150 hours, the AIML487 students are expected to have more work done, and have some novelty and original contributions in the research project. Since the number of students in AIML487 is small in 2021, the project can be handled case by case. The course coordinator should be consulted if an examiner is uncertain regarding the grade. 

\chapter{Presentation Day}

The presentation day is an opportunity for students to demonstrate
their oral presentation skills. The primary objective of the
presentation day is to prepare students for the real-world, where
presentations are an integral component of business.  This will be a
all day event which is usually scheduled on the last day of exams.
There will be one or two Dean's sessions - to which industry will be
invited, students will be selected for these sessions based on their
presentations at the start of Trimester 2, and their submitted
report.  This is a serious opportunity for your work to be seen on a
larger stage, and perhaps lead to some new opportunities.

\section{Overview}

The presentations will each be 15 minutes long in total. This should break
down into around 10 minutes of speaking, 3 minutes for questions and 2
minutes for change over. Strict time-keeping will be followed, and
presentations that run over the time limit will be cut short. This is highly
undesirable and does not auger well for a good presentation grade.

You should expect to get through at most seven slides. Any more, and you will
be speaking far too quickly to give an effective presentation. Make sure that
you practice your talk several times to get the timing right.

The talk should cover all aspects of your project, including the motivation,
problem statement, discussion approach, technical aspects of approach and
experimental results. The following suggestion is one possible outline, though
naturally you should vary the structure to suit the specifics of your project.
\\

\begin{tabular}{l|l}
Slide & Title\\
\hline
Slide 1 & Title, Name and Supervisor Name(s)\\
Slide 2 & Introduction + Motivation\\
Slide 3 & Problem Statement and Discussion of Possible Approaches\\
Slide 4 & Overview + Justification of Chosen Approach\\
Slide 5 & Experimental Results and/or Findings\\
Slide 6 & Contribution\\
Slide 7 & Conclusion\\
\end{tabular}
\\

{\bf NOTE:} The format for presentations should be either in PDF or
PowerPoint.  Presentations will need to be submitted the day before, so we can make sure
they're all loaded on the presentation machines. We will \emph{not} check that
your files work correctly, so you should do that yourself.

\section{Demonstration}

Most students will be able to provide a sufficient illustration of
their project during the presentation.  However, in some cases, a
demonstration of the working artifact may be preferred. Think carefully about
this; a demonstration may seem like a good idea, but they can easily break the
flow of a talk and detract from the message being delivered. It is very easy to
have the audience looking curiously at your project rather than listening to
you speak! Videos of your project can be more effective for this
reason - and are strongly recommended as live demonstrations are
inherently high risk and it is not unusual for them to go wrong.

{\bf NOTE:} The course coordinator and appropriate technical
staff must be notified well before the presentation day if a student wishes
to use a demonstration.

\section{Assessment}
The examiners will consider the presentations according to the following criteria:

\begin{itemize}
\item Motivation (i.e. was the project properly motivated?)
\item Research Statement (i.e. was the problem being addressed clearly
identified?)
\item Methodology (i.e. how you conducted your research?)
\item Implementation (i.e. was a sensible discussion of what has been done provided?)
\item Evaluation Approach (i.e. was the approach being taken clearly identified?)
\item Justification of Evaluation (i.e. was the evaluation approach justified?)
\item Results (i.e. are results presented in a clear manner?)
\item Professionalism (i.e. was the presentation of a professional nature?)
\item Structure (i.e. was the presentation structured appropriately?)
\end{itemize}

\noindent {\bf NOTE:} There is limited time within the presentation and, hence,
we do not expect you will cover all of the above in detail.

%\input{assessment}


%%%%%%%%%%%%%%%%%%%%%%%%%%%%%%%%%%%%%%%%%%%%%%%%%%%%%%%

%%%%%%%%%%%%%%%%%%%%%%%%%%%%%%%%%%%%%%%%%%%%%%%%%%%%%%%


%\bibliographystyle{ieeetr}
%\bibliographystyle{acm}
%\bibliography{sample}


\end{document}
