\chapter{Presentation Day}

The presentation day is an opportunity for students to demonstrate
their oral presentation skills. The primary objective of the
presentation day is to prepare students for the real-world, where
presentations are an integral component of business.  This will be a
all day event which is usually scheduled on the last day of exams.
There will be one or two Dean's sessions - to which industry will be
invited, students will be selected for these sessions based on their
presentations at the start of Trimester 2, and their submitted
report.  This is a serious opportunity for your work to be seen on a
larger stage, and perhaps lead to some new opportunities.

\section{Overview}

The presentations will each be 15 minutes long in total. This should break
down into around 10 minutes of speaking, 3 minutes for questions and 2
minutes for change over. Strict time-keeping will be followed, and
presentations that run over the time limit will be cut short. This is highly
undesirable and does not auger well for a good presentation grade.

You should expect to get through at most seven slides. Any more, and you will
be speaking far too quickly to give an effective presentation. Make sure that
you practice your talk several times to get the timing right.

The talk should cover all aspects of your project, including the motivation,
problem statement, discussion approach, technical aspects of approach and
experimental results. The following suggestion is one possible outline, though
naturally you should vary the structure to suit the specifics of your project.


\begin{tabular}{l|l}
Slide & Title\\
\hline
Slide 1 & Title, Name and Supervisor Name(s)\\
Slide 2 & Introduction + Motivation\\
Slide 3 & Problem Statement and Discussion of Possible Approaches\\
Slide 4 & Overview + Justification of Chosen Approach\\
Slide 5 & Experimental Results and/or Findings\\
Slide 6 & Contribution\\
Slide 7 & Conclusion\\
\end{tabular}


{\bf NOTE:} The format for presentations should be either in PDF or
PowerPoint.  Presentations will need to be submitted the day before, so we can make sure
they're all loaded on the presentation machines. We will \emph{not} check that
your files work correctly, so you should do that yourself.

\section{Demonstration}

Most students will be able to provide a sufficient illustration of
their project during the presentation.  However, in some cases, a
demonstration of the working artifact may be preferred. Think carefully about
this; a demonstration may seem like a good idea, but they can easily break the
flow of a talk and detract from the message being delivered. It is very easy to
have the audience looking curiously at your project rather than listening to
you speak! Videos of your project can be more effective for this
reason - and are strongly recommended as live demonstrations are
inherently high risk and it is not unusual for them to go wrong.

{\bf NOTE:} The course coordinator and appropriate technical
staff must be notified well before the presentation day if a student wishes
to use a demonstration.

\section{Assessment}
The examiners will consider the presentations according to the following criteria:

\begin{itemize}
\item Motivation (i.e. was the project properly motivated?)
\item Research Statement (i.e. was the problem being addressed clearly
identified?)
\item Methodology (i.e. how you conducted your research?)
\item Implementation (i.e. was a sensible discussion of what has been done provided?)
\item Evaluation Approach (i.e. was the approach being taken clearly identified?)
\item Justification of Evaluation (i.e. was the evaluation approach justified?)
\item Results (i.e. are results presented in a clear manner?)
\item Professionalism (i.e. was the presentation of a professional nature?)
\item Structure (i.e. was the presentation structured appropriately?)
\end{itemize}

\noindent {\bf NOTE:} There is limited time within the presentation and, hence,
we do not expect you will cover all of the above in detail.
