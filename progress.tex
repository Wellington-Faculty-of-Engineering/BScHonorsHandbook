\chapter{Preliminary Report}
At the conclusion of the first trimester, students are expected to
submit a preliminary report which outlines the progress they have made,
and identifies any outstanding issues where feedback is required. This
report should be considered a first step towards the final report -
including a good treatment of the introduction and related/background
work.  However, as a primary purpose of the preliminary report is to give the examination
committee the opportunity to comment on the student's progress (and
identify any areas of concern), it will also include sections on work
done, requests for feedback, and a revised timeline.

{\bf Please Note:} The preliminary report grade is used as part of the final grade 
if it helps increase the overall grade.

\section{Suggested Organisation}
A sensible outline for the preliminary report is as follows:
\begin{itemize}
\item {\bf Introduction / Proposal Review.}  This should briefly
  outline the project and if necessary reevaluate the original plan in light of
  what has been learned in the interim.  In particular, any significant
  deviations in the research problem being addressed
  should be clearly highlighted and justified.
\item {\bf Literature review}.  This should discuss any existing
  solutions to the given problem, and may reference academic papers,
  books and other sources as appropriate.  Care should be taken to
  identify key differences between these solutions, and that being
  developed in the project.
\item {\bf Work Done.}  This should discuss what progress has been
  made.  In many cases, the evaluation proper will not yet have begun.  However,
  it is important to demonstrate that sufficient thought has been
  given to the evaluation.
\item {\bf Future Plan.}  This should highlight the main components
  which remain to be done, and provide a proposed time-line in which
  this will happen.  In putting together a time line, students must
  take into account upcoming examinations, coursework deadlines and
  other disruptions.
\item {\bf Request for Feedback.}  This should highlight any
  difficulties currently faced, and make specific requests for
  guidance from the examination committee.  For example, a student may
  be unsure how best to evaluate their artifact, and would appreciate
  suggestions for alternative methods.
\end{itemize}

The report does not have to confirm to the above structure.
For example, in some cases, students may wish to present preliminary
experimental results, or include a more detailed literature survey, 
particularly if they are doing the 45 point course and so the report
is two thirds of the way through the project rather than halve way.\\

\noindent {\bf NOTE:} in the event of an aegrotat application, the preliminary
report may be used (in conjunction with the snapshot submission) as guidance 
for final grade.

\section{Getting help with writing}
Students struggling with writing and presentation should seek help from the student 
learning support as early as possible.
{\footnotesize \url{http://www.victoria.ac.nz/st_services/slss/}}.


\section{Format}
The following points clarify the main requirements of the preliminary
report:
\begin{itemize}
\item The report should be written using the ECS report templates provided 
on the resources page of the course site.
(available for latex and MS Word).
\item The report is expected to be around 8 pages in length. As a rough
breakdown, a page of introduction and three to four pages on
background/related work.  An additional page each on progress
and future plans would be appropriate. Longer (or shorter) reports are
permitted, but students are advised to ensure all necessary detail is provided.
\item The report should be written in such a way that any 4th year student in your specialisation
  can understand.  Since the report will be assessed by an independent
  examiner (i.e. not just the supervisor), it is critical that all
  examiners can properly understand what has been achieved.
\item The report should include the original project proposal as an
  appendix.
\end{itemize}
Finally, the preliminary report must be submitted via the \href{https://apps.ecs.vuw.ac.nz/submit/}{online
  submission system} on or before the given due date.


\section{Assessment Process}

The preliminary report will be read by two examiners, one of
which is the primary supervisor. Students are required to give a 5 minute
presentation at a specialisation meeting of the primary supervisor.
Constructive feedback should be given after the presentation.
We may record the sessions, so students can reference feedback.


