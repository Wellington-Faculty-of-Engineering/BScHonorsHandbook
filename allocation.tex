\chapter{Project Allocation}
The first stage of the BSc Hons project is the
allocation of projects to students.  This process attempts to allocate
students to the projects they prefer.  Indeed, it is in the interests
of both students and staff that this is done as accurately, and
quickly as possible.  Once the allocation is complete, students must
produce a project proposal in conjunction with their
supervisor(s).

\section{Choosing a Project}

Projects are listed on the \href{https://ecs.wgtn.ac.nz/Courses/COMP489_2022FY/Projects}{COMP 489 Projects page} linked from the combined course page currently listed under COMP 489. There will be a link on this page for the form to submit your preferences for projects.  
You should generally pick projects you have talked to the supervisor about.

We cannot guarantee that every student will be allocated to a project they
prefer.  In the unlikely event of a student being allocated to a
project that they believe is not suitable for them, they should
immediately contact the course coordinator.

\section{Proposal}

Once the allocation of students to projects is complete, students are
expected to meet with their supervisors and put together a {\em project proposal}.

\noindent {NOTE:} it is the student's responsibility to contact their
supervisor and arrange an appropriate meeting time.\\
{Students are required to submit a report (no more than four pages) for the proposal stage by the end of week 5. Generally,
the report should include the following topics:}

\begin{itemize}
    \item Title Page
    \item An overview of the research problem being addressed by the project.
    \item A high-level summary of state-of-the-art techniques to solve this problem.
    \item A statement of key motivations, including limitations or issues that the current/state-of-the-art methods have and this project is to tackle.
    \item A statement of the overall goal and specific objectives, hypotheses, or research questions.
    \item A statement regarding the proposed method to investigate to the problem.
    \item A statement regarding the proposed evaluation method, e.g. possible available data and the evaluation measures.
    \item A discussion of any ethical considerations around the project.
    \item A statement, if applicable, regarding any budgetary requirements, including
      appropriate justification.
    \item A statement regarding any risks or hazards that the project
      poses (either in the development itself, or in using the final
      artifact).
    \item A discussion of any other requirements for the project to be successfully
    completed. This might be access to particular equipment or rooms, special IP
    issues etc.
    \item Provide a plan B to continue working on the project if Covid 19 restrictions cause limited access to equipment, environment, or users.
\item References and Bibliography
\end{itemize}

The project may include
\begin{itemize}
	\item Table of Contents and Glossary
    \item Provide a proposed project time line, in the form of a Gantt chart (or similar).
\end {itemize}

There is the potential to provide some funding for some projects.  
The funding is primarily to help purchase items
necessary for the project, although it can be used for other purposes
(e.g. as koha for user-experiments or surveys).  Students must justify their
budgetary requirements in the proposal report.

For external projects, it is the norm that the sponsoring organisation fund any related costs for the project. Any exceptions will need an approval from the Head of School.

\subsection{Assessment Process}

Constructive feedback should be given two weeks after the proposal submission deadline.  
The examiners are expected to read the report and give feedback to the supervisors. The aim of this process is to identify:
\begin{enumerate}
    \item whether the project is viable and sensible for the given specialisation;
    \item whether there are any obvious issues which must be addressed.
\end{enumerate} 
Where necessary, some comments will be communicated
to the student by the supervisor.

\section{Intellectual Property Agreement}

All students working with external partners are required to submit a signed intellectual property agreement along with their {proposal report}.  The purpose of the intellectual property agreement is simply to identify the stakeholders in the project.
