\chapter{Project Allocation}
The first stage of the BSc Hons project is the
allocation of projects to students.  This process attempts to allocate
students to the projects they prefer.  Indeed, it is in the interests
of both students and staff that this is done as accurately, and
quickly as possible.  Once the allocation is complete, students must
produce a project proposal in conjunction with their
supervisor(s).

\section{Choosing a Project}

The online \href{https://ecs.wgtn.ac.nz/apps/projectselection/}{Project Allocation System (PAS)} is used by both staff and
students to register and rank projects.  Prior to the start of
Trimester 1, staff upload descriptions of the projects
they wish to supervise. The PAS system contains only a brief description of
each projects. During pick week students are encouraged to speak to potential
supervisors to gain a better idea of what is involved.

The algorithm we use for matching students to staff and projects is a
variation on the Deferred Acceptance Algorithm (DAA) -- specifically
we use a simplified version of the the North American Medical
Placement system which allocates about 20 thousand students to
internships every year.  The nice thing about this algorithm is that
it produces stable matches and is strategy proof - that is,
misleading the system about your rankings (e.g., to try and get a better
allocation) will only lead to you getting a worse outcome.

Once everyone's rankings are complete, we run the algorithm and we're
done.  Well, almost.  Unfortunately there may be left over students and
projects where the algorithm expended all the students choices without
finding them a project (because the supervisors they chose were
fully allocated to other students).  In this case we enter what is
called the {\em scramble} - which really means we just assign the remainders manually.

\noindent There are several important points to make about the PAS system:
\begin{itemize}
\item {\bf Students cannot pick more than two projects (excluding the industry projects) from any given (primary) supervisor}.  If you do this, you will get an error message and the system will not add your selection.  If you wish to change your project selection, you will need to remove one of your previous choices first.   This helps to ensure that student preferences
  are diverse, and do not single out specific supervisors.  For
  example, without this restriction, a given student may only select
  projects from one supervisor, hoping to ensure they are allocated
  that supervisor.  However, if several students adopt this strategy
  for the same supervisor, then a problem arises as each supervisor
  may only take on a limited number of students (typically 1 or 2
  students).

\item {\bf Students must rank at least seven different projects}  If you wish
to alter your project rankings (otherwise it is in order you
added them to your list) just drag and drop the projects in your list to reflect
your preference order.  Once you have picked at least 7 projects you
will be able to use the submit button to register your choices.  If
you pick less than 7, the submit button will not be displayed and when we run the algorithm, you will go directly to the scramble
(see above).  This means that everyone else will get their choices
before you.

\item {\bf Staff rank the student-project selections}. Each primary
  supervisor for a project you have ranked, will in turn rank your
  selection against all other selections by other students. This
  ranking will include consideration into your suitability for any
  specific project, along with the supervisor's own preference for
  that project (we limit the number of projects a supervisor can be
  allocated, given those limits, they may prefer to have project widget allocated over
  project gadget).

\item {\bf Privacy} All student rankings and staff rankings are kept
  private.  Academic staff will \textbf{not} see student rankings, and students will not see staff rankings.  Therefore you can feel free to rank your most preferred projects without fear of offending a staff member.

\item {\bf Some projects have co-supervisors listed}.  Depending on
  the particular staff involved, some supervisors will share equally
  in supervision, whereas others may choose to have a co-supervisor
  who can provide additional expertise for a project - but play a
  lesser role in the supervision as a whole.  Usually, all administrative aspects of the project are the responsibility of the primary supervisor.

\item {\bf Industry projects}. Industry projects have an academic staff listed as either primary or secondary supervisor. However, the academic staff will be the student's first point of contact in relation to all aspects of the project. \\
A student can pick as many industry projects from a supervisor and are not counted as part of \textbf{"maximum of two projects"} constraint discussed above.\\

\textbf{Please Note:} If an BSc Hon student chooses an industry project, we must make additional checks that the project has a research focus. 

\end{itemize}
Unfortunately, despite all of these recommendations, we cannot
guarantee that every student will be allocated to a project they
prefer.  In the unlikely event of a student being allocated to a
project that they believe is not suitable for them, they should
immediately contact the course coordinator.

\section{Proposal}

Once the allocation of students to projects is complete, students are
expected to meet with their supervisors and put together a {\em
  project proposal}.\\

\noindent {NOTE:} it is the student's responsibility to contact their
supervisor and arrange an appropriate meeting time.\\
{Students are required to submit a report (no more than four pages) for the proposal stage by the end of week 4. Generally,
the report should include the following topics:}

\begin{itemize}
\item An overview of the research problem being addressed by the project.
\item A high-level summary of state-of-the-art techniques to this problem
\item A statement of key motivations, including limitations or issues that the current/state-of-the-art methods have and this project is to tackle
\item A statement of the overall goal and specific objectives, hypotheses, or research questions
\item A statement regarding the proposed method to investigate to the problem.
\item A statement regarding the proposed evaluation method, e.g. possible available data and the evaluation measures
\item A discussion of any ethical considerations around the project.
\item A statement, if applicable, regarding any budgetary requirements, including
  appropriate justification.
\item A statement regarding any risks or hazards that the project
  poses (either in the development itself, or in using the final
  artifact).
\item A discussion of any other requirements for the project to be successfully
completed. This might be access to particular equipment or rooms, special IP
issues etc.
\item Provide a proposed project time line, in the form of a Gantt
  chart (or similar).
\end{itemize}


{Proposal pages not included in the page limit:}
\begin{itemize}
	\item Title Page
	\item Table of Contents and Glossary
	\item References and Bibliography
	\item Project Gantt Chart
\end {itemize}
%Because we have a 5 minute time limit for the oral presentation, you
% should to distil the above points into 4 content slides.  Ideally, one
% slide on: the problem you are solving, your proposed
%solution or direction, your proposed evaluation method(s),
% Ethics/Risks/Budget, and finally a timeline mapping out major
%milestones.  The ethics/risk/budget slide does not count as one of the
%content slides.  You will need to submit these slides by the due date,
%and then use them in your oral presentation in the following week.

A small amount of funding is available for every project (the exact
amount depends on the specialisation, and should be clarified by the
course coordinator).  The funding is primarily to help purchase items
necessary for the project, although it can be used for other purposes
(e.g. as koha for user-experiments or surveys).  Students must justify their
budgetary requirements in the proposal report.

For industry projects, it is a norm that the industry sponsor funds any related costs for the project. Any exceptions will need an approval from the Head of School.

\subsection{Assessment Process}

{Constructive feedback should be given two weeks after the report submission deadline.  The examiners are expected to read the report and give feedback to the supervisors.}

The aim of this process is to identify:
firstly, whether the project is viable and sensible for the given
specialisation; secondly, whether there are any obvious issues which
must be addressed.  Where necessary, some comments will be communicated
to the student by the supervisor.

\section{Intellectual Property Agreement}

All students working with industry partners are required to submit a signed intellectual property agreement along with their {proposal report}.  The purpose of the intellectual property agreement is simply to identify those parties who are stakeholders in the project.
