\chapter{Introduction}
The ENGR/COMP/ELCO489, and CGRA489 project courses consist of an individual project
done under the supervision of one (or more) academic staff.  Projects
are also offered in partnership with industry - in which case
supervision is shared with an industry supervisor.     The underlying
aim of the project is to show-case the skills learnt during your degree, and to
demonstrate your independent and critical thinking.  The project will involve
designing, implementing and evaluating a solution to a complex engineering
problem (ENGR489) or research problem (COMP489, CGRA489, ELCO489).  You
will present a series of written reports on your project, and conclude with
an oral presentation that may include a practical demonstration (where
appropriate).

\section{Aims and Scope}
The aim of this document is to provide a comprehensive guide to the ENGR/COMP
/ELCO489, and CGRA489 courses, for both students and staff. In
particular, the document sets out the requirements of the course and clarifies the way in which student projects 
will be assessed and supervised.

\section{Engineering versus Science}
An important consideration is the distinction between the project
courses taken as part of the BE and those taken as part of the BSc(Hon)
or postgraduate diploma.
The former requires students to undertake a suitable {\em engineering} project,
whilst the latter require a suitable {\em research} project.  There are many
similarities between these two types of project, but there are also some
important differences.

This document will highlight the differences between what is expected for an
\engr project, and that of a \bsc project.  In summary, the main differences
are:

\begin{itemize}
\item \engr projects are expected to solve real-world problems using
  technically innovative solutions.  \engr projects must show an
  emphasis on design and provide evidence of the effectiveness of the devised
  solutions through appropriate evaluation. Students are expected to
  demonstrate craftsmanship in the design and implementation of their solution,
  and to use engineering processes and/or notations appropriate for their
  specialisation.

\item \bsc projects are based on planning and implementation of an experimental protocol, and collect, 
analyse and interpret data through
  research. Such projects should aim to make novel contributions to
  the academic research literature but not necessary.  \bsc students are expected to
  demonstrate mathematical rigour (where appropriate), and use
  scientific experimentation to make critical observations. The literature
  survey for \bsc projects will typically draw on research papers in journals
  and conferences.
\end{itemize}

{\bf NOTE:} Students should consult with their supervisor(s) and/or
the course coordinator if they are unsure as to whether their project
is an appropriate \engr, \bsc project.



\section{Design, Implement and Evaluate}

A typical project can be thought of as designing,
implementing and evaluating an {\em artifact}.  The term artifact
refers to that which is delivered by the project, and may represent
something concrete (such as an electromechanical device) or
something more abstract (e.g. a mathematical proof or a taxonomy).  
In more detail, the waterfall methodology has
three main stages:
\begin{itemize}
\item{\bf Design.}  This is the process of taking a problem and
  devising a suitable solution by considering the various options
  available.  One may design a concrete artifact, such as a software
  or hardware system.  Or, the design component of a project may be
  less tangible.  For example, designing an experiment to make some
  crucial observations about an existing system. \engr projects in particular
  should investigate multiple possible solutions so that engineering
  tradeoffs can be discussed.
\item{\bf Implementation.}  This is the process of taking a given
  design and fleshing out the details to the point where a working
  system forms.  Considerable skill is often required to use
  appropriate tools and techniques to make this happen.  For example,
  software development practices, such as testing, will be necessary
  to deliver a working software system.  Likewise, constructing an
  electrical circuit board may be a necessary step in delivering a
  hardware system.
\item{\bf Evaluation.}  This is the process of reflecting on the
  artifact produced, primarily for the purpose of demonstrating it is
  ``good'' in some sense.  For example, consider a tool for finding
  software bugs.  Important questions to answer here include: {\em
    Does the tool find all possible errors?} {\em How long does the
    tool take to find errors?} {\em What are the tool's capabilities and limitations?}  
Such questions are typically answered
  through experimental observation of the artifact in operation.
\end{itemize}
Finally, it should be noted that there is no formal requirement to
undertake these stages in any given order.  For example, software
development processes, such as agile or XP, dictate a more fluid
approach.  Nevertheless, these components should still be evident
within the project.

\pagebreak
\section{Project Timeline}
The following provides a rough overview of the project timeline, and identifies the main points of interest.
\begin{center}
\begin{tabular}{|l|l|}
\hline
Week 1 & {Students rank projects using project allocation system.}\\
Week 2 & Project allocation performed by course coordinator.\\
Week 3 & Students meet with supervisor(s) and begin work.\\
Week 4 (27th March, 5pm)& \textbf{{Students submit project proposals and IP forms on ECS Wiki}}\\
week 5 & Work continues; students meet regularly with supervisors.\\
...&  $\ldots$\\
Week 12 Monday (1 June)& \textbf{ENGR489 Students submit their preliminary report.}\\
\hline
\hline
Mid-Year Break& Work continues around examinations.\\
&Students meet with supervisors where possible.\\
\hline
\hline
Week 1 & {Students give presentation on preliminary report (Not Compulsory)}\\
...& Work continues; students meet regularly with supervisors.\\
Week 7 (end of)& {Students submit a draft of their final report to their Supevisors.}\\
Week 10 & {ENGR489 students submit a project snapshot. Not assessed at this point.}\\
Week 12 (end of)& \textbf{{Students submit final report.} {Students submit presentation slides.}}\\
\hline
\hline
First Friday of T2 Exam Period & \textbf{Students present their work during conference day.}\\
\hline
\end{tabular}
\end{center}
