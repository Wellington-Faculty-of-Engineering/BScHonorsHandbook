\chapter{Supervision and You}

%% This text is based on the regulations for PhD supervision from
%% FGR.  This has been simplified and repackaged for Hons.

As this is likely your first experience with an individual supervised project, it
can be difficult to calibrate your expectations against your
supervisor's.  This section aims to describe what you
should be getting from your supervisor, and what your supervisor
should be getting from you.  If your experience under supervision differs widely from
the guidelines given in this chapter, you should first discuss this with
your supervisor and if it cannot be resolved to your satisfaction
please bring your query to the BSc Hons coordinator.  The earlier issues are
identified and resolved, the better things will be handled.  There is
little we can do to resolve long standing problems a few weeks out
from submission.

\section{Supervisors Responsibilities}

It is the responsibility of your supervisors to guide you through the
academic requirements of your project. Supervisors will:

\begin{itemize}
\item meet with you regularly
\item provide you with academic guidance and scholarly direction
\item assess your progress and give you feedback
\item act as guides to University facilities
\item make sure you comply with the University’s administrative regulations.
\end{itemize}

\subsection{Regular Meetings}

It is expected that you and your supervisor will meet in person
regularly and individually:

\begin{itemize}
\item For a project supervised by a VUW staff member we would expect 
a 30 minute individual meeting each week.  You may agree
with your supervisor to different arrangements that better suit the
nature of the project, but it important to have regular meetings.

\item For a project supervised by an external supervisor and VUW
supervisor, we would expect a fortnightly supervision meeting with the
external supervisor.  The VUW supervisor would not be expected to
attend every meeting, although may, but would generally have a combined meeting at
least every month.
\end{itemize}

\subsection{Academic Guidance}

Your supervisors in conjunction with the research seminars and course coordinators 
will provide guidance on a range of academic matters. These include:

\begin{itemize}
\item the standards required for an honours project
\item planning your research
\item skills you will need to acquire
\item research resources
\item methodology
\item undertaking a literature review
\item ethical, legal, professional and safety issues
\end{itemize}

Throughout, your supervisors will bear in mind the expectations of examiners.

\subsection{Assessing Your Progress and Feedback}

Supervisors will assess your progress and provide you with constructive feedback throughout your project. 
They will need to ensure that you possess the understanding and abilities to:

\begin{itemize}
\item carry out your project as envisioned
\item complete your work on time, meeting the various deadlines for assessment.
\item Provide prompt feedback on your work.  The university guideline
  for feedback is 3 weeks, ECS aims for a 2 week turnaround.
\end{itemize}

\subsection{Support}

Your supervisor(s) is also there to support you. If you encounter
problems of any kind, you should feel free to discuss them with them -
especially if it could have an impact on your project work.  The BSc Hons 
coordinator is also available to help and offer support in such
situations, especially if you are not comfortable discussing matters
directly with your supervisor.  If they can't help, they will be able
to direct you to various student support services run by the
University -- a guidline to these services will be linked from the
course homepage.

\section{Your Responsibilities as a Project Student}

You will need to abide by the University regulations governing your
degree.

\subsection{Planning and Actively Pursuing Your Work}

You have an obligation to devote sufficient time to your work, to
complete each phase on time, and to avoid activities that interfere
with your satisfactory and timely completion of the project.  
For the 30 point courses: CGRA489, COMP489 and ELCO489
they are 15 points per trimester, thus you should expect to spend 
on average 10 hours per week on your project,
spread over the 30 weeks that the course runs (i.e. including
mid-trimester breaks, and the mid-year break).  
For the 45 point course AIML487, you will likely have one semester with 
more work, thus in your semester with only 2 other courses you are likely to work 20 hours per week.
It can be quite challenging to maintain steady progress and dedicate the time as
course loads increase during the trimester, however, it is important
that you manage your time well so that you can devote time each week to the project.  
You will get little benefit from your
supervisor if you treat your project as a series of short term crunches.
Certainly they will not be able to provide timely feedback or
appropriate guidance in this situation.

\subsection{Ethics}

It is expected that you conduct your research in an ethical
manner.  We already have ethical approval for standard
user interface testing for software Ethics \#29386 linked from the course resources page
Additionally you must:
\begin{itemize}
\item where appropriate, discuss ethics with your supervisors
\item familiarise yourself with the \href{https://www.wgtn.ac.nz/research/support/ethics}{University’s ethical guidelines}
\item obtain approval from the relevant ethics committee for work involving human or animal subjects.
\end{itemize}
The link to Ethics application process is also available on the BSc Hons Wiki.

It is also important to conduct yourself ethically in your project in relation to 
academic misconduct.  Breaches of the plagiarism or other forms of misconduct
will be treated very seriously. If you are concerned about plagiarism you 
should talk to your supervisor or the course coordinator to clarify the 
situation.

\subsection{Safety \& Health}
The university's approach to health and safety is based on risk management. 
There is a significant strengthening of level of responsibility for students and supervisors. 
Students must discuss with supervisors and show in the project proposal report 
(due at the end of week 5) that they have identified safety risks and developed a plan to manage them.

\noindent Students are expected to be aware of the Health and Safety at Work Act 2015. See : \linebreak  {\footnotesize\url{http://www.business.govt.nz/worksafe/hswa}}.

Students need to discuss with their supervisors and fill out the health and safety 
plan available on the ECS Wiki. They need to fill 'ECS Project Information Form' 
and 'Project Description and Safety Plan'. A sample can be found on the Wiki.

Please Note: For any work that takes place off VUW campuses, the students need fill 
'ECS Off Campus Activity Plan'. Please contact Roger Cliffe for the form.

ALL filled Health and Safety forms must be emailed to \href{mailto://ecs-safety@ecs.vuw.ac.nz}{ecs-safety@ecs.vuw.ac.nz}.
