\chapter{Introduction}
The BSc Hons (COMP/ELCO/CGRA489 and AIML487) project courses consist of an individual or small 
group project
done under the supervision of one (or more) academic staff.  Some projects
are also offered in partnership with industry - in which case
supervision is shared with an industry supervisor.  The underlying
aim of the project is to develop your research ability and demonstrate your independent  
and critical thinking.  The project will involve refining a project proposal, reviewing and analysing relevant literature, potentially developing an artifact, and evaluating the artifact or research problem.  
You will present a final written report, and conclude with
an oral presentation.

\section{Aims and Scope}
The aim of this document is to provide a comprehensive guide to the BSc Hons  project, for both students and staff. In
particular, the document sets out the requirements of the course and clarifies the way in which student projects 
will be assessed and supervised.

\section{Engineering versus Science}
Many of the students in 400 level courses are doing ENGR489, that is an engineering project which is more about implementation and process, while this handbook describes the
\textbf{research} projects.

Research projects at 4th year range from literature reviews and analysis, repeating existing published work, through to the testing of new artifacts, and creation of knowledge.  The first part of the project is to define the scope of the project and questions that are being answered.
If an artifact is created then there needs to be an experimental protocol to collect, 
analyse and interpret data. BSc Hons students are expected to
  demonstrate mathematical rigour (where appropriate), and use
  scientific experimentation to make critical observations. The literature
  survey for the projects will typically draw on research papers in journals
  and conferences.

{\bf NOTE:} If unsure you should consult with their supervisor(s) and/or
the course coordinator to ensure that the project is a research project.

\section{Plan, Create, Test and Evaluate}

A project can be thought of as planning, reviewing the research area, typically creating an artifact based on research,
and evaluating the {\em artifact}.  The term artifact
refers to student created software, theoretical framework, taxonomy, dataset or other work apart from than the report created as part of the project. 

\pagebreak
\section{Project Timeline}
The following provides a rough overview of the project timeline, and identifies the main points of interest.
\begin{center}
\begin{tabular}{|l|p{10cm}|}
\hline
Week 1 & Students review projects and meet with potential supervisors\\
Week 2 & Discussion with potential supervisors continues\\
Week 3 & Projects are allocated, and student meet with their allocated supervisor\\
Week 4 & Students meet with supervisor(s) and begin work\\
Week 5 & \textbf{ students submit project proposals} \href{https://apps.ecs.vuw.ac.nz/submit/COMP489}{ECS submission system}\\
week 6 & Work continues; students meet regularly with supervisors.\\
...&  $\ldots$\\
Week 12 & \textbf{Students submit their progress report.} This is worth 20\% if it increases the final grade.\\
\hline
\hline
Mid-Year Break& Work continues around examinations.\\
&Students meet with supervisors where possible.\\
\hline
\hline
Week 1 & {Students give a presentation to their associated research group}\\
...    & Work continues; students meet regularly with supervisors.\\
Week 7 (end of)& {Students submit a draft of their final report to their Supervisors.}\\
Week 12 (end of)& \textbf{Students submit final report.}\\
\hline
\hline
T2 Exam Period & \textbf{Students present their work during presentation day.}\\
\hline
\end{tabular}
\end{center}
